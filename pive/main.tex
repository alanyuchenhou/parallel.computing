\documentclass[12pt]{article}
\renewcommand*{\familydefault}{\sfdefault}
\usepackage{listings}
\usepackage{amsmath}
\usepackage{fullpage}
\usepackage{tabularx}
\usepackage{graphicx}
\begin{document}
\title{Cpt S 483 Assignment Cover Sheet \\ Fall 2014
}
\author{Yuchen Hou}
\maketitle

(To be turned in along with each homework and program project submission)

Assignment: project 5

For individual assignments:

Student name (Last, First):

For team projects:

List of all students (Last, First):

List of collaborative personnel (excluding team participants):

I certify that I have listed above all the sources that I consulted regarding this assignment, and that I have not received or given any assistance that is contrary to the letter or the spirit of the collaboration guidelines for this assignment. (print name here if using a word processor).

Assignment Project Participant(s): Hou, Yuchen;

Today's Date: \today

\pagebreak

\section{Runtime}
The total runtime and pi value for different thread counts and input sizes is listed below:
\lstinputlisting{data.csv}

\section{Speedup}
\begin{figure}%% [htb]
  \centering
      {\includegraphics[width=1\linewidth]{data.png}} \rule{1\linewidth}{1pt}
      \caption{Speedup for various thread counts and input sizes. Log scale is set on x axe to show the large range of the data.}
      \label{fig:data}
\end{figure}
The speedup is shown in Figure \ref{fig:data}. Both 2 thread and 4 thread execution achieve significant speedup for large enough input sizes. The speedup for 2 thread execution exceeds that of 1 thread execution at input size 4k and then approaches maximum 2 at input size 64k. The speedup for 4 thread execution exceeds that of 2 tread execution at input size 1M and then approaches maximum 3.5 at size 671M. Although the speedup for 4 thread execution has fluctuations, the 2 speedup lines are both reasonable.

\end{document}
