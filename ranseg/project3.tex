\documentclass[12pt]{article}
\renewcommand*{\familydefault}{\sfdefault}
\usepackage{listings}
\usepackage{amsmath}
\usepackage{fullpage}
\usepackage{tabularx}
\usepackage{graphicx}
\begin{document}
\title{Cpt S 483 Assignment Cover Sheet \\ Fall 2014
}
\author{Armen Abnousi, Yuchen Hou}
\maketitle

(To be turned in along with each homework and program project submission)

Assignment: project 3

For individual assignments:

Student name (Last, First):

For team projects:

List of all students (Last, First):

List of collaborative personnel (excluding team participants):

I certify that I have listed above all the sources that I consulted regarding this assignment, and that I have not received or given any assistance that is contrary to the letter or the spirit of the collaboration guidelines for this assignment. (print name here if using a word processor).

Assignment Project Participant(s): Hou, Yuchen; Abnousi, Armen

Today's Date: \today

\pagebreak

\section{Process count analysis}
\begin{figure}[htb]
  \centering
      {\includegraphics[width=1\linewidth]{speedup.png}} \rule{1\linewidth}{1pt}
      \caption{The speedup for fixed input size (100M) and various process count with logscaling for both x and y axes.}
      \label{fig:process}
\end{figure}
The speedup for fixed input size and various process count is shown in Figure \ref{fig:process}, and the speedup achieved is almost linear with respect to process count.
\section{Input size analysis}
\begin{figure}[htb]
  \centering
      {\includegraphics[width=1\linewidth]{input.png}} \rule{1\linewidth}{1pt}
      \caption{The runtime for various input size and fixed process count (16) with logscaling for both x and y axes.}
      \label{fig:input}
\end{figure}
The runtime for various input size and fixed process count is shown in Figure \ref{fig:input}. We ran the identical experiments multiple times to determine the suitable range of input size. 
The data we collected tells us than in smaller input sizes the runtime doesn't change so much, we can be an indicator of dominant communication time over computation time.
For input sizes greater than $10^9$, the slope of increase in runtime is much sharper than previous experiments which can mean that it's over-utilizing the system. (This input size is too big for this number of processes).

\end{document}
