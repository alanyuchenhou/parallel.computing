\documentclass[12pt]{article}
\renewcommand*{\familydefault}{\sfdefault}
\usepackage{listings}
\usepackage{amsmath}
\usepackage{fullpage}
\usepackage{tabularx}
\usepackage{graphicx}
\begin{document}
\title{Cpt S 483 Assignment Cover Sheet \\ Fall 2014
}
\author{Yuchen Hou}
\maketitle

(To be turned in along with each homework and program project submission)

Assignment: homework4

For individual assignments:

Student name (Last, First):

For team projects:

List of all students (Last, First):

List of collaborative personnel (excluding team participants):

I certify that I have listed above all the sources that I consulted regarding this assignment, and that I have not received or given any assistance that is contrary to the letter or the spirit of the collaboration guidelines for this assignment. (print name here if using a word processor).

Assignment Project Participant(s): Hou, Yuchen

Today's Date: \today

\pagebreak

\section{Algorithm}
An efficient parallel counting sort algorithm is expressed in the form of pseudocode below.
\begin{enumerate}
  \item processor[i] sorts its local array A[i] using counting sort;\\
    the frequency of integer k in A[i] is stored as local-frequency[k];
  \item an allgather communication sends every local-frequency from processor[i] to all other processors;
  \item every processor reduces the local-frequencies to get gloabel-frequency: global-frequency[k] = sum of every local-frequency[k];
  \item every processor calculates the global-start-index for integer k: global-index = prefixsum(global-frequency);
  \item every processor calculates the destination processor for integer k (the processor that should store integer k): destination[k] = global-index[k]/(n/p);
  \item processor[i] partitions A[i] into p parts where part[j] contains integers in A[i] with destination = j;
  \item an alltoallv communication sends part[j] in A[i] to processor[j];
  \item processor[j] sorts all the part[j]s it received into a sorted array B[j];
\end{enumerate}
The final sorted array is the concatenation of B[j].

\section{Time complexity}
The time complexity is shown in Table \ref{tab:complexity}.
\begin{table}[htb]
  \centering
  \begin{tabularx}{\textwidth}{|l|l|X|} \hline
    step & computation time & communication time \\ \hline
    1 & $O(\frac{n}{p}+c)$ & 0  \\ \hline
    2 & 0 & $O(\tau p + \mu p^2)$ \\ \hline
    3 & $O(c)$ & 0 \\ \hline
    4 & $O(c)$ & 0 \\ \hline
    5 & $O(c)$ & 0 \\ \hline
    6 & $O(\frac{n}{p})$ & 0 \\ \hline
    7 & 0 & $O(\tau p + \mu \frac{n}{p})$ \\ \hline
    8 & $O(\frac{n}{p}+c)$ & 0 \\ \hline
    total & $O(\frac{n}{p}+c)$ & $O(\tau p + \mu (p^2+\frac{n}{p}))$ \\ \hline
  \end{tabularx}
  \caption{The parallel time complexity analysis of counting sort}
  \label{tab:complexity}
\end{table}


\end{document}
